% !TeX program = xelatex
\documentclass{ctexart}
\usepackage{../../../templates/note-template}

\title{函数极限与连续笔记}
\date{\today}

\begin{document}
\maketitle

\section{函数概念与特性}

\begin{theorem}[函数]
    我们只研究单值函数,即对于每一个 $x$,函数 $y=f(x)$ 有唯一确定的值的函数
\end{theorem}

\begin{theorem}[反函数]
    把函数$y=f(x)$的x和y互换位置(定义域和值域也互换),得到新的函数$y=f^{-1}(x)=\dfrac{1}{f(x)}$称为$f(x)$的反函数    
    \begin{enumerate}
        \item 由于$y=f(x)$和$y=f^{-1}(x)$都是单值函数,故对于每一个 $x$,函数 $y$ 有唯一确定的值,对于每一个 $y$,函数 $x$ 也有唯一确定的值。又因严格单调函数一定是一一对应的函数,而一一对应的函数不一定是单调函数,故:\\
        “是严格单调函数”$\Rightarrow$“是一一对应的函数”$\Leftrightarrow$“有反函数”
        \item $f^{-1}(f(x))=x$,$f(f^{-1}(x))=x$
        \item $y=f(x)$和$x=f^{-1}(y)$是同一个函数,在同一坐标系上的图像也完全重合,$y=f(x)$和$y=f^{-1}(x)$才互为反函数且因x,y互换,函数图像关于$y=x$对称
        \item $y=f(x)$和$y=f^{-1}(x)$在区间内的单调性相同,且在交点处的函数值相等
    \end{enumerate}
\end{theorem}

\begin{example}[3个双曲函数]反双曲正弦函数:$y=\ln(x+\sqrt{x^2+1})$,双曲正弦函数:$y=\dfrac{e^x-e^{-x}}{2}$,双曲余弦函数:$y=\dfrac{e^x+e^{-x}}{2}$
    \begin{itemize}
        \item 定义域都是$(-\infty,+\infty)$
        \item $y=\ln(x+\sqrt{x^2+1})$和$y=\dfrac{e^x-e^{-x}}{2}$互为反函数,都是奇函数,且都单调递增
        \item $y=\dfrac{e^x+e^{-x}}{2}$是偶函数,于y轴相交于点$(0,1)$
    \end{itemize}
    \textbf{函数图像如下:}

    \begin{center}
        \begin{tikzpicture}[scale=1]
        % 左侧图:双曲正弦和反双曲正弦
        \begin{scope}[xshift=-3.5cm]
        % 坐标轴
        \draw[->] (-3,0) -- (3,0) node[right] {$x$};
        \draw[->] (0,-3) -- (0,3) node[above] {$y$};
        
        % y=x参考线
        \draw[gray, dashed, thin] (-2,-2) -- (2,2);
        
        % 双曲正弦函数 sinh(x) = (e^x - e^(-x))/2
        \begin{scope}
        \clip (-2.5,-3) rectangle (2.5,3);
        \draw[blue, smooth, thick, domain=-2.5:2.5, samples=200] 
        plot (\x, {(exp(\x) - exp(-\x))/2});
        \end{scope}
        
        % 反双曲正弦函数 arcsinh(x) = ln(x + sqrt(x^2 + 1))
        \draw[red, smooth, thick, domain=-2.5:2.5, samples=200] 
        plot (\x, {ln(\x + sqrt(\x*\x + 1))});
        \end{scope}

        % 右侧图:双曲余弦
        \begin{scope}[xshift=3.5cm]
        % 坐标轴
        \draw[->] (-3,0) -- (3,0) node[right] {$x$};
        \draw[->] (0,-3) -- (0,3) node[above] {$y$};
            
        % 双曲余弦函数 cosh(x) = (e^x + e^(-x))/2
        \begin{scope}
        \clip (-2.5,-3) rectangle (2.5,3);
        \draw[green!60!black, smooth, thick, domain=-2.5:2.5, samples=200] 
        plot (\x, {(exp(\x) + exp(-\x))/2});
        \end{scope}
            
        % 最小点
        \draw[fill] (0,1) circle[radius=1.5pt] node[below right] {$(0,1)$};
        \end{scope}

        % 图例放在底部中间
        \node[anchor=north, inner sep=1pt] 
        at (0,-3.5) {
        \begin{tabular}{c@{\hspace{2em}}c@{\hspace{2em}}c@{\hspace{2em}}c}
        \bfseries\textcolor{blue}{\rule[0.5ex]{1em}{0.4pt}} $\dfrac{e^x-e^{-x}}{2}$ &
        \bfseries\textcolor{red}{\rule[0.5ex]{1em}{0.4pt}} $\ln(x+\sqrt{x^2+1})$ &
        \bfseries\textcolor{green!60!black}{\rule[0.5ex]{1em}{0.4pt}} $\dfrac{e^x+e^{-x}}{2}$ &
        \bfseries\textcolor{gray}{\rule[0.5ex]{1em}{0.4pt} \tiny dashed} $y=x$
        \end{tabular}
        };
        \end{tikzpicture}
    \end{center}
    \textbf{重要结论}
    \begin{enumerate}
        \item $x\Rightarrow0$时,$y=\ln(x+\sqrt{x^2+1}) \sim x$
        \item $[ln(x+\sqrt{x^2+1})]'=\dfrac{1}{\sqrt{x^2+1}}$,于是$\int \dfrac{1}{\sqrt{x^2+1}}dx=\ln(x+\sqrt{x^2+1})+C$
        \item 由于$y=\ln(x+\sqrt{x^2+1})$是奇函数,故$\int_{-1}^1[\ln(x+\sqrt{x^2+1})+x^2]dx=\int_{-1}^1x^2dx=\dfrac{2}{3}$
    \end{enumerate}
\end{example}

\begin{definition}[复合函数]
    有一函数$y=f(u)$,$u \in D_f$,另有一函数$u=f(x)$,$x \in D_g$满足:
    \begin{enumerate}
        \item $u=f(x)$的值域含于在$y=f(u)$的定义域内,即$g(x) \subseteq D_f$
        \item $u=f(x)$的定义域$D_g$不做要求
    \end{enumerate}
    则:
    \begin{enumerate}
        \item $y=f[g(x)]$,$x \in D_g$为由$y=f(u)$和$u=f(x)$构成的复合函数
        \item $f[g(x)]$的值域含于在$y=f(u)$的值域内$f[g(x)]\subseteq f(u)$
    \end{enumerate}
\end{definition}

\begin{example}[隐函数]
    方程$F(x,y)=0$在某区间能满足单值函数的要求,则$F(x,y)=0$在上述区间内确定了一个隐函数$y=y(x)$,求$y(x_0)$时,如好求则直接求出,若不好求则用观察法,例如:
    \begin{enumerate}
        \item 设函数$y=y(x)$由方程$lny-\dfrac{x}{y}+x=0$确定,则$y(2)=1$
        \item 设函数$y=y(x)$由方程$lny+e^{y-1}=\dfrac{x}{2}$确定,则$y(2)=1$
    \end{enumerate}
\end{example}

\begin{theorem}[函数的四种特性]
    要熟练将数学含义和代数式相互转化\\
        \textbf{有界性}\\
        $y=f(x)$,$x \in I$有界$\Leftrightarrow$存在$M>0$,使得$|f(x)|\leq M$,$x \in I$
            \begin{enumerate}
                \item 不知区间,无法讨论有界性
                \item 只要在区间I上或其端点处存在点$x_0$,使得$\lim\limits_{x \to x_0}f(x)=\infty$,则$f(x)$无界(举反例)
                \item 证明有界的思路,给$f(x)$套绝对值,利用各种不等式
                    \begin{enumerate}
                        \item 三角函数不等式:$|\sin x|\leq 1$,$|\cos x|\leq 1$,$|\sin x+\cos x|\leq \sqrt{2}$
                        \item 指数函数不等式:$e^x>1+x$(当$x\neq 0$时)
                        \item 对数函数不等式:$\ln(1+x)<x$(当$x>0$时)
                        \item 绝对值不等式:$|a+b|\leq |a|+|b|$,$||a|-|b||\leq |a-b|$
                        \item 基本不等式:$\dfrac{a+b}{2}\geq \sqrt{ab}$(当$a,b>0$时)
                        \item 柯西不等式:$(a_1^2+a_2^2+...+a_n^2)(b_1^2+b_2^2+...+b_n^2)\geq (a_1b_1+a_2b_2+...+a_nb_n)^2$
                    \end{enumerate}
            \end{enumerate}        
        \textbf{单调性}\\
        对于任意$x_1$,$x_2 \in D$,$x_1\neq x_2$
            \begin{flushleft} 
                \begin{tabular}{|c|c|c|c|}
                \hline
                \makecell{单调性} & \makecell{基础定义} & \makecell{高级定义} & \makecell{导数} \\
                \hline
                \makecell{单调\\递增} & \makecell{$x_1<x_2$\\$\Rightarrow$\\$f(x_1)<f(x_2)$} & \makecell{$(x_1-x_2)[f(x_1)-f(x_2)]>0$} & \makecell{$f'(x)>0$} \\
                \hline
                \makecell{单调\\递减} & \makecell{$x_1<x_2$\\$\Rightarrow$\\$f(x_1)>f(x_2)$} & \makecell{$(x_1-x_2)[f(x_1)-f(x_2)]<0$} & \makecell{$f'(x)<0$} \\
                \hline
                \makecell{单调\\不减} & \makecell{$x_1<x_2$\\$\Rightarrow$\\$f(x_1)\leq f(x_2)$} & \makecell{$(x_1-x_2)[f(x_1)-f(x_2)]\geq 0$} & \makecell{$f'(x)\geq 0$} \\
                \hline
                \makecell{单调\\不增} & \makecell{$x_1<x_2$\\$\Rightarrow$\\$f(x_1)\geq f(x_2)$} & \makecell{$(x_1-x_2)[f(x_1)-f(x_2)]\leq 0$} & \makecell{$f'(x)\leq 0$} \\
                \hline
                \end{tabular}
                \end{flushleft}
        \textbf{奇偶性}\\
            \begin{enumerate}
                \item 基础定义:若定义域关于原点对称,且
                    \begin{enumerate}
                        \item 对于任意$x$,有$f(-x)=-f(x)\Leftrightarrow y=f(x)$是奇函数$\Leftrightarrow y=f(x)$关于原点对称$\Rightarrow y=f(0)=0$
                        \item 对于任意$x$,有$f(-x)=f(x)\Leftrightarrow y=f(x)$是偶函数$\Leftrightarrow y=f(x)$关于y轴对称
                        \item 对于任意$x$,有$f(-x)=\pm f(x)\Leftrightarrow y=f(x)$是奇偶函数$\Leftrightarrow y=f(x)$关于原点且关于y轴对称
                    \end{enumerate}
                \item 重要结论
                    \begin{itemize}
                        \item $f(x)+f(-x)$必是偶函数,如$y=\dfrac{e^x+e^{-x}}{2}$
                        \item $f(x)-f(-x)$必是奇函数,如$y=\dfrac{e^x-e^{-x}}{2}$
                        \item $f(x)f(-x)$必是偶函数
                        \item $ln[f(x)/f(-x)]$必是奇函数
                    \end{itemize}
                \item 复合函数$f[g(x)]$的奇偶性(内偶则偶,内奇同外)
                    \begin{itemize}
                        \item 奇$[$偶$]$ $\Rightarrow$ 偶,如$sinx^2$
                        \item 偶$[$奇$]$ $\Rightarrow$ 偶,如$cos(sinx)$,$|sinx|$
                        \item 奇$[$奇$]$ $\Rightarrow$ 奇,如$sin\dfrac{1}{x}$,$\sqrt[3]{tanx}$
                        \item 偶$[$偶$]$ $\Rightarrow$ 偶,如$cos|x|$,$|cosx|$
                        \item 非奇非偶$[$偶$]$ $\Rightarrow$ 偶,如$e^{x^2}$,$ln|x|$
                    \end{itemize}
                \item 特殊函数的奇偶性:$[ln(x+\sqrt{x^2+1})]'=\dfrac{1}{\sqrt{x^2+1}}$ (奇变偶)
                \item 导数的奇偶性:求导后奇偶互变
                \item 积分的奇偶性,奇函数积分变偶函数
                \item 高级奇偶性判别式:设对任意的x,y,都有$f(x+y)=f(x)+f(y)$,则$f(x)$是奇函数
                    \begin{itemize}
                        \item 证明思路:先$f(x+0)=f(x)+f(0)$,得证$f(0)=0$,后$f(x-x)=f(x)+f(-x)=0$,得证$f(-x)=-f(x)$
                    \end{itemize}
            \end{enumerate}
        \textbf{周期性}\\
        $y=f(x)$是周期函数$\Leftrightarrow$存在$T>0$,使得$f(x+T)=f(x)$,若$g(x)$以T为周期,则:
            \begin{itemize}
                \item $g(ax+b)$以$\dfrac{T}{|a|}$为周期
                \item 复合函数$f[g(x)]$一定是周期函数,如$e^{sinx}$,$cos^2x$
                \item $g'(x)$依旧是以T为周期的周期函数
                \item 只有当$\int_{0}^{T}g(x)dx=0$时,$\int_{0}^{x}g(t)dt$才是以T为周期的周期函数
            \end{itemize}
\end{theorem}

\section{函数的图像}
\begin{example}[基本初等函数]反对幂指三$+$常数函数\\
\textbf{幂函数}\\
当$x>0$且$a\neq -1$时,$y=x^a$全都单调递增,具有于$y=x^a=x^{-1}=\dfrac{1}{x}$相反的单调性,又因$lnx$也在$x>0$时也单调递增,故:
\begin{itemize}
    \item 见到$\sqrt{u}$,$\sqrt[3]{u}$时,可用$u$来研究最值
    \item 见到时$|u|$,由$|u|=\sqrt{u^2}$,可用$u^2$来研究最值
    \item 见到$u_1 u_2 u_3$时,可用$ln(u_1 u_2 u_3)=lnu_1+lnu_2+lnu_3$来研究最值
    \item 见到$\dfrac{1}{u}$时,可用$u$来研究最值(结论相反)
\end{itemize}



    
\end{example}

\end{document}