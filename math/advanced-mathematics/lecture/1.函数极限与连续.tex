% !TeX program = xelatex
\documentclass{ctexart}
\usepackage{../../../templates/note-template}

\title{函数极限与连续错题集}
\date{\today}

\begin{document}
\maketitle

\section{函数概念与特性}

\begin{theorem}[函数]
    我们只研究单值函数,即对于每一个 $x$,函数 $y=f(x)$ 有唯一确定的值的函数
\end{theorem}

\begin{theorem}[反函数]
    把函数$y=f(x)$的x和y互换位置(定义域和值域也互换),得到新的函数$y=f^{-1}(x)$称为$f(x)$的反函数    
    \begin{enumerate}
        \item 由于$y=f(x)$和$y=f^{-1}(x)$都是单值函数,故对于每一个 $x$,函数 $y$ 有唯一确定的值,对于每一个 $y$,函数 $x$ 也有唯一确定的值。又因严格单调函数一定是一一对应的函数,而一一对应的函数不一定是单调函数,故:\\
        “是严格单调函数”$\rightarrow$“是一一对应的函数”$\Leftrightarrow$“有反函数”
        \item $f^{-1}(x)f(x)=x$,$f(x)f^{-1}(x)=x$
        \item $y=f(x)$和$x=f^{-1}(y)$是同一个函数,在同一坐标系上的图像也完全重合,$y=f(x)$和$y=f^{-1}(x)$才互为反函数且因x,y互换,函数图像关于$y=x$对称
        \item $y=f(x)$和$y=f^{-1}(x)$在区间内的单调性相同,且在交点处的函数值相等
    \end{enumerate}
\end{theorem}

\begin{example}[3个双曲函数]反双曲正弦函数:$y=\ln(x+\sqrt{x^2+1})$,双曲正弦函数:$y=\dfrac{e^x-e^{-x}}{2}$,双曲余弦函数:$y=\dfrac{e^x+e^{-x}}{2}$
    \begin{itemize}
        \item 定义域都是$(-\infty,+\infty)$
        \item $y=\ln(x+\sqrt{x^2+1})$和$y=\dfrac{e^x-e^{-x}}{2}$互为反函数,都是奇函数,且都单调递增
        \item $y=\dfrac{e^x+e^{-x}}{2}$是偶函数,于y轴相交于点$(0,1)$
    \end{itemize}
    \textbf{函数图像如下:}

    \begin{center}
        \begin{tikzpicture}[scale=1]
        % 左侧图:双曲正弦和反双曲正弦
        \begin{scope}[xshift=-3.5cm]
        % 坐标轴
        \draw[->] (-3,0) -- (3,0) node[right] {$x$};
        \draw[->] (0,-3) -- (0,3) node[above] {$y$};
        
        % y=x参考线
        \draw[gray, dashed, thin] (-2,-2) -- (2,2);
        
        % 双曲正弦函数 sinh(x) = (e^x - e^(-x))/2
        \begin{scope}
        \clip (-2.5,-3) rectangle (2.5,3);
        \draw[blue, smooth, thick, domain=-2.5:2.5, samples=200] 
        plot (\x, {(exp(\x) - exp(-\x))/2});
        \end{scope}
        
        % 反双曲正弦函数 arcsinh(x) = ln(x + sqrt(x^2 + 1))
        \draw[red, smooth, thick, domain=-2.5:2.5, samples=200] 
        plot (\x, {ln(\x + sqrt(\x*\x + 1))});
        \end{scope}

        % 右侧图:双曲余弦
        \begin{scope}[xshift=3.5cm]
        % 坐标轴
        \draw[->] (-3,0) -- (3,0) node[right] {$x$};
        \draw[->] (0,-3) -- (0,3) node[above] {$y$};
            
        % 双曲余弦函数 cosh(x) = (e^x + e^(-x))/2
        \begin{scope}
        \clip (-2.5,-3) rectangle (2.5,3);
        \draw[green!60!black, smooth, thick, domain=-2.5:2.5, samples=200] 
        plot (\x, {(exp(\x) + exp(-\x))/2});
        \end{scope}
            
        % 最小点
        \draw[fill] (0,1) circle[radius=1.5pt] node[below right] {$(0,1)$};
        \end{scope}

        % 图例放在底部中间
        \node[anchor=north, inner sep=1pt] 
        at (0,-3.5) {
        \begin{tabular}{c@{\hspace{2em}}c@{\hspace{2em}}c@{\hspace{2em}}c}
        \bfseries\textcolor{blue}{\rule[0.5ex]{1em}{0.4pt}} $\dfrac{e^x-e^{-x}}{2}$ &
        \bfseries\textcolor{red}{\rule[0.5ex]{1em}{0.4pt}} $\ln(x+\sqrt{x^2+1})$ &
        \bfseries\textcolor{green!60!black}{\rule[0.5ex]{1em}{0.4pt}} $\dfrac{e^x+e^{-x}}{2}$ &
        \bfseries\textcolor{gray}{\rule[0.5ex]{1em}{0.4pt} \tiny dashed} $y=x$
        \end{tabular}
        };
        \end{tikzpicture}
    \end{center}
    \textbf{重要结论}
    \begin{enumerate}
        \item $x\rightarrow0$时,$y=\ln(x+\sqrt{x^2+1}) \sim x$
        \item $[ln(x+\sqrt{x^2+1})]'=\dfrac{1}{\sqrt{x^2+1}}$,于是$\int \dfrac{1}{\sqrt{x^2+1}}dx=\ln(x+\sqrt{x^2+1})+C$
        \item 由于$y=\ln(x+\sqrt{x^2+1})$是奇函数,故$\int_{-1}^1[\ln(x+\sqrt{x^2+1})+x^2]dx=\int_{-1}^1x^2dx=\dfrac{2}{3}$
    \end{enumerate}
\end{example}

\end{document}