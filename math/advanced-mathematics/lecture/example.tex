% !TeX program = xelatex
\documentclass{ctexart}
\usepackage{../../../templates/note-template}

\title{函数极限与连续性}
\date{\today}

\begin{document}
\maketitle

\section{函数概念与特性}
函数极限是微积分中最基本的概念之一,它描述了函数在某一点附近的行为特征。

\begin{definition}[函数极限]
设函数 $f(x)$ 在点 $x_0$ 的某去心邻域内有定义,如果存在常数 $A$,对于任意给定的正数 $\varepsilon$,总存在正数 $\delta$,当 $0<|x-x_0|<\delta$ 时,有:
\[ |f(x)-A|<\varepsilon \]
则称 $A$ 为 $f(x)$ 当 $x \to x_0$ 时的极限,记为:
\[ \lim_{x \to x_0} f(x) = A \]
\end{definition}

这个定义虽然看起来复杂,但本质上描述的是函数值无限接近某个常数的过程。

\section{极限的性质}
极限的四则运算法则是解决极限问题的基本工具,掌握这些性质可以大大简化计算过程。

\begin{theorem}[极限四则运算]
设 $\lim\limits_{x \to x_0} f(x) = A$,$\lim\limits_{x \to x_0} g(x) = B$,则:
\begin{enumerate}
    \item $\lim\limits_{x \to x_0} [f(x) \pm g(x)] = A \pm B$
    \item $\lim\limits_{x \to x_0} [f(x) \cdot g(x)] = A \cdot B$
    \item $\lim\limits_{x \to x_0} \frac{f(x)}{g(x)} = \frac{A}{B}$ (当 $B \neq 0$)
\end{enumerate}
\end{theorem}

这些性质告诉我们,在计算极限时,可以将复杂的表达式分解成简单的部分分别计算。

\section{重要极限}
在极限计算中,有几个特别重要的极限,它们构成了解决复杂极限问题的基础。

\begin{example}[第一重要极限及其应用]
\textbf{基本形式:}
\[ \lim_{x \to 0} \frac{\sin x}{x} = 1 \]

\textbf{相关变形:}
\begin{itemize}
    \item $\lim\limits_{x \to 0} \frac{\tan x}{x} = 1$
    \item $\lim\limits_{x \to 0} \frac{\arcsin x}{x} = 1$
    \item $\lim\limits_{x \to 0} \frac{1-\cos x}{x^2} = \frac{1}{2}$
\end{itemize}

\textbf{使用注意:}
\begin{itemize}
    \item $x$ 必须以弧度为单位
    \item 分母必须是 $x$
    \item 常与等价无穷小代换结合使用
\end{itemize}

\textbf{几何意义:}当 $x \to 0$ 时:
\begin{itemize}
    \item $\sin x \sim x$
    \item $\tan x \sim x$
    \item $1-\cos x \sim \frac{x^2}{2}$
\end{itemize}
\end{example}

\begin{example}[第二重要极限及其应用]
\textbf{基本形式:}
\[ \lim_{x \to \infty} (1+\frac{1}{x})^x = e \]

\textbf{常见变形:}
\begin{itemize}
    \item $\lim\limits_{n \to \infty} (1+\frac{1}{n})^n = e$
    \item $\lim\limits_{x \to 0} (1+x)^{\frac{1}{x}} = e$
    \item $\lim\limits_{x \to 0} \frac{\ln(1+x)}{x} = 1$
\end{itemize}

\textbf{应用技巧:}
\begin{itemize}
    \item 注意提取极限中的幂指数形式
    \item 将复杂表达式转化为标准形式
    \item 结合对数进行化简
\end{itemize}

\textbf{易错点:}
\begin{itemize}
    \item 底数和指数的变化需同步考虑
    \item 注意变形时的等价替换
    \item 区分极限不存在的情况
\end{itemize}
\end{example}

\begin{example}[等价无穷小的应用]
\textbf{常用等价无穷小($x \to 0$):}
\begin{itemize}
    \item $\sin x \sim x$
    \item $\tan x \sim x$
    \item $\arcsin x \sim x$
    \item $\ln(1+x) \sim x$
    \item $1-\cos x \sim \frac{x^2}{2}$
    \item $(1+x)^\alpha-1 \sim \alpha x$
\end{itemize}

\textbf{使用原则:}
\begin{enumerate}
    \item 乘除可以替换
    \item 加减需要相同阶数
    \item 替换后不能出现未定式
\end{enumerate}
\end{example}

\section{连续函数的性质}
\begin{theorem}[连续函数的性质]
设函数 $f(x)$ 在闭区间 $[a,b]$ 上连续,则:
\begin{enumerate}
    \item 有界性:$f(x)$ 在 $[a,b]$ 上有界
    \item 最值定理:$f(x)$ 在 $[a,b]$ 上必取得最大值和最小值
    \item 介值定理:对于任意介于最小值和最大值之间的值,函数都能取到
\end{enumerate}
\end{theorem}

\end{document}