% !TeX program = xelatex
\documentclass{ctexart}
\usepackage{../../../templates/note-template}

\title{函数极限与连续错题集}
\date{\today}

\begin{document}
\maketitle

\section{函数的概念与特性}

\begin{strategy}[求函数]
\textbf{题型特征:}
\begin{itemize}
    \item 给出自变量为x变形式的函数表达式,要求求出自变量为x的函数表达式
    \item x变形式的函数表达式可能有一个或两个
    \item x变形式可能是$-x$,$\dfrac{1}{x}$,$x+\dfrac{1}{x}$等形式
\end{itemize}

\textbf{解题思路:}
\begin{enumerate}
    \item x变形式的函数表达式有一个
    \begin{itemize}
        \item 想办法构造出复合函数所表达的式子,可利用换元法
    \end{itemize}
    
    \item x变形式的函数表达式有两个
    \begin{itemize}
        \item 两个x变形式互换得新的函数表达式(一般互为相反数或倒数)
        \item 与原函数表达式构成方程组,求解$f(x)$
    \end{itemize}
    
\end{enumerate}

\textbf{易错点总结:}
\begin{itemize}
    \item 没有解题思路惯性思维
    \item 未能构造出复合函数所表达的式子
    \item 未能正确互换两个x变形式
    \item 未能正确构造出方程组
    \item 未能正确求解$f(x)$(计算能力)
\end{itemize}
\end{strategy}

\newproblem

\begin{problem}[单x变形式求函数]
设 $f\left(x+\dfrac{1}{x}\right) = \dfrac{x+x^{3}}{1+x^{4}}$,则当$x\geqslant2$时,$f(x)=\underline{\qquad}$.

\end{problem}

\begin{solution}

\textbf{解答步骤:}
\begin{enumerate}
    \item \important{关键点}:想办法构造出$x+\dfrac{1}{x}$所表达的式子
    
    \item \textbf{详细过程:}
    \[ \begin{aligned}
    &f\left(x+\dfrac{1}{x}\right) = \dfrac{x+x^{3}}{1+x^{4}}\ \text{,令}\ x+\dfrac{1}{x}=t\ \text{,则}\ x^{2}+1=tx\ \text{,}\\
    &\text{代入}\ f(t)=\dfrac{tx^{2}}{(tx)^{2}-2x^{2}}\ \text{,得}\ f(t)=\dfrac{t}{t^{2}-1}\ \text{,}\\
    &\text{即}\ f(x)=\dfrac{x}{x^{2}-1}\ \text{。}
    \end{aligned} \]
    
\end{enumerate}


\textbf{易错提示:}
\begin{itemize}
    \item 没有解题思路惯性思维
    \item 换元法不熟练
\end{itemize}

\end{solution}

\newproblem

\begin{problem}[双x变形式求函数]
计算下列极限:设函数 $f(x)$ 的定义域为 $(0,+\infty)$ ,且满足 $2f(x)+x^2f\left(\dfrac{1}{x}\right)=\dfrac{x^2+2x}{\sqrt{1+x^2}}$ ,则 $f(x)=\underline{\qquad}$.

\end{problem}

\begin{solution}

\textbf{解答步骤:}
\begin{enumerate}
    \item \important{关键点}:$f(x)$与$f\left(\dfrac{1}{x}\right)$互换,得方程组求解f(x)
    
    \item \textbf{详细过程:}
    \[ \begin{aligned}
    &\text{由}\ 2 f(x)+x^{2} f\left(\dfrac{1}{x}\right)=\dfrac{x^{2}+2 x}{\sqrt{1+x^{2}}}\ \text{,得} \\
    &2 f\left(\dfrac{1}{x}\right)+\dfrac{1}{x^{2}} f(x)=\dfrac{\dfrac{1}{x^{2}}+\dfrac{2}{x}}{\sqrt{1+\dfrac{1}{x^{2}}}}
    =\dfrac{\dfrac{1+2 x}{x^{2}}}{\dfrac{\sqrt{1+x^{2}}}{x}}=\dfrac{1+2 x}{x \sqrt{1+x^{2}}}\ \text{,}\\
    &\text{结合以上两式,消去}\ f\left(\dfrac{1}{x}\right)\ \text{,可得}\ f(x)=\dfrac{x}{\sqrt{1+x^{2}}}\ \text{。}
    \end{aligned} \]
    
\end{enumerate}


\textbf{易错提示:}
\begin{itemize}
    \item 认真计算,避免代数运算错误
        \begin{enumerate}
            \item 消去$f\left(\dfrac{1}{x}\right)$时,只让一个式子做变换,减少出错概率
            \item $\dfrac{1}{x^2}+\dfrac{2}{x}=\dfrac{1+2x}{x^2}\neq\dfrac{2+x}{x^2}$
            \item 计算到$\dfrac{3}{2}f(x)=\dfrac{1}{\sqrt{1+x^2}}\cdot\dfrac{3x}{2}$时,不要丢单独计算出来的$\dfrac{3x}{2}$的$x$
        \end{enumerate}
\end{itemize}

\end{solution}

\end{document}