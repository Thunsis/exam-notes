% !TeX program = xelatex
\documentclass{ctexart}
\usepackage{../../../templates/note-template}

\title{极限与连续错题集}
\date{\today}

\begin{document}
\maketitle

\section{函数极限}
在解决极限问题时,我们常常遇到需要使用有理化方法的情况。这类题目通常涉及根式,且直接代入会得到未定式。掌握有理化求极限的方法,对于解决此类问题至关重要。

\begin{strategy}[有理化求极限]
\textbf{题型特征:}
\begin{itemize}
    \item 含有根式:$\sqrt{1+x}$、$\sqrt[n]{1+x}$ 等
    \item 为 $\frac{0}{0}$ 型未定式
    \item 分子含根式,分母为多项式
\end{itemize}

\textbf{解题思路:}
\begin{enumerate}
    \item 识别特征
    \begin{itemize}
        \item 是否含有根式
        \item 代入后是否为未定式
        \item 分子分母结构分析
    \end{itemize}
    
    \item 关键方法
    \begin{itemize}
        \item 有理化:乘以分子分母的共轭式
        \item 约分:化简后的分子分母约去公因式
        \item 代值:最后代入极限点求值
    \end{itemize}
    
    \item 计算技巧
    \begin{itemize}
        \item 提取公因式简化计算
        \item 使用等价无穷小代换
        \item 注意正负号变化
    \end{itemize}
\end{enumerate}

\textbf{易错点总结:}
\begin{itemize}
    \item 直接代入导致 $\frac{0}{0}$ 型未定式
    \item 有理化时忘记同分母运算
    \item 约分过程符号或计算错误
    \item 遗漏分母不为零的条件讨论
\end{itemize}
\end{strategy}

\newproblem
在下面这个典型例题中,我们将详细展示有理化方法的应用过程。这个例子代表了一类最基本的有理化求极限问题,通过这个例子可以很好地理解有理化的思路和技巧。

\begin{problem}[有理化求极限]
计算下列极限:
\[ \lim_{x \to 0} \frac{\sqrt{1+x}-1}{x} \]

\textbf{分析:} 代入 $x=0$ 得到 $\frac{0}{0}$ 型未定式,需要通过有理化处理。
\end{problem}

\begin{solution}
解决这类问题时,关键在于找到合适的有理化方式。通常我们选择与分子共轭的式子进行有理化,这样可以消去根式。下面是详细的解答过程:

\textbf{解答步骤:}
\begin{enumerate}
    \item \important{关键点}:分子分母同乘 $\sqrt{1+x}+1$
    
    \item \textbf{详细过程:}
    \[ \begin{aligned}
    & \frac{\sqrt{1+x}-1}{x} \cdot \frac{\sqrt{1+x}+1}{\sqrt{1+x}+1} \\[5pt]
    & = \frac{(\sqrt{1+x})^2-1}{x(\sqrt{1+x}+1)} \\[5pt]
    & = \frac{1+x-1}{x(\sqrt{1+x}+1)} \\[5pt]
    & = \frac{1}{\sqrt{1+x}+1}
    \end{aligned} \]
    
    \item \textbf{求极限:}
    \[ \lim_{x \to 0} \frac{1}{\sqrt{1+0}+1} = \frac{1}{2} \]
\end{enumerate}

通过上述解答过程,我们可以看到有理化方法的优雅之处:它能够将含有根式的极限问题转化为简单的代数运算。这种方法不仅适用于本题,还可以推广到更复杂的情况。

\textbf{易错提示:}
\begin{itemize}
    \item 直接代入会得到 $\frac{0}{0}$ 型未定式
    \item 有理化时要注意分子分母同时乘以共轭式
    \item 化简时注意不要遗漏分母项
\end{itemize}

这个例子展示了有理化求极限的基本思路和常见陷阱。在解决类似问题时,要特别注意有理化的选择和运算过程的严谨性。
\end{solution}

\end{document}